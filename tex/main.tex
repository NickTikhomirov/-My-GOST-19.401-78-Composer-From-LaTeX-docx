% !TEX program = xelatex

\documentclass[12pt,a4paper]{article}


% все пакеты

% Куча различных необычных матриц, дробей и прочего
% Советую заглянуть в документацию по пакету и почитать
\usepackage{amsmath}
% Некоторые математические шрифты вроде mathfrac, mathcal и mathsf 
\usepackage{amsfonts}

% Пакет, позволяет писать код латеха не только в ASCII
\usepackage[utf8]{inputenc}
% Пакет с языками
\usepackage[english, russian]{babel}

% Дополнительные возможности по выравниванию
% Не уверен, что это нужно
\usepackage{ragged2e}

\usepackage{esint}
\usepackage{systeme}
\usepackage[nointegrals]{wasysym}
\usepackage{graphicx}
\usepackage{bbold}
\usepackage{xcolor}
\usepackage{hyperref}
\usepackage{varwidth}
\usepackage{ifthen}
\usepackage{marginnote}
\usepackage{multirow}
\usepackage{setspace}
\usepackage{icomma}
\usepackage{extsizes}
\usepackage{anyfontsize}
\usepackage{environ}


% Пакет на команду, которая проверяет, что на странице достаточно места для контента
% Используется для более гибкого переноса на новую страницу
\usepackage{needspace}

% Без этого пакета первый абзац каждой главы идёт без красной строки
% Почему? Потому что так решил латех 
\usepackage{indentfirst}

% Пакет для настройки параметров заголовков (цвет, размер, выравнивание...)
\usepackage[center]{titlesec}


\usepackage{pdfpages}



% Настройки шрифтов, фича xelatex
\usepackage{fontspec} % пакет для выбора шрифта
\defaultfontfeatures{Ligatures={TeX},Renderer=Basic}  %% свойства шрифтов по умолчанию
\setmainfont[Ligatures={TeX,Historic}]{Times New Roman} %% задаёт основной шрифт документа
\setsansfont{Montserrat}                    %% задаёт шрифт без засечек

\setmonofont{Fira Code} % очень много итальянской магии от итальянских волшебников


% Отсутпы по краям
\usepackage[left=3cm,right=1cm,top=2cm,bottom=2cm,bindingoffset=0cm]{geometry}

% адреса для поиска картинок
\graphicspath{ {../img/}{./img/}{./tex/img/} }



\setlength{\parindent}{1.25cm} % абзац






\usepackage{needspace}
\usepackage{lscape}
\usepackage{verbatimbox}
\usepackage{indentfirst}
\usepackage{titletoc}
\usepackage[center]{titlesec}
\usepackage{xpatch,letltxmacro}
\usepackage{threeparttable}
\usepackage{tabularx}
\usepackage{longtable}
\usepackage{float}
\usepackage{multicol}
\usepackage{bold-extra}
\usepackage{listings}
\usepackage{fancyhdr}
\pagestyle{fancy}
\fancyhead{} 
\fancyhead[C]{\thepage \\ МГТУ.111111.001-01 12 01}
\fancyfoot{}
\renewcommand{\footrulewidth}{0 mm}
\renewcommand{\headrulewidth}{0 mm}

% Размер заголовка с одной циферкой
\newcommand{\sectionsize}{\fontsize{14pt}{14pt}}


\makeatletter
\renewcommand\tableofcontents{%
    \centerline{\textbf{СОДЕРЖАНИЕ}}\vspace{-1cm}

    \null\hfill\textbf{\sectionsize}\hfill\null\par
    \@mkboth{\MakeUppercase\contentsname}{\MakeUppercase\contentsname}%
    \@starttoc{toc}%
}
\makeatother

\dottedcontents{section}[1.6em]{}{0.8cm}{.3pc}
\dottedcontents{subsection}[3.1em]{}{0.8cm}{.3pc}


\titleformat{\section}
{\centering\bfseries\MakeUppercase}
{\sectionsize\selectfont\thesection{}. }
{0cm}
{\sectionsize\selectfont}




\lstset{language=[Sharp]C,
  showspaces=true,
  showtabs=false,
  breaklines=true,
  showstringspaces=false,
  %breakatwhitespace=true,
  escapeinside={(*@}{@*)},
  basicstyle=\ttfamily,
  columns=fullflexible
}



\newcommand{\getcode}[1]{

\newpage

\section{Текст файла #1.}\stepcounter{section}


\begingroup
\itshape

\lstinputlisting{../cs/#1}

\endgroup

}





\begin{document}


\includepdf
[
    pages={1}, 
    %pagecommand={
    %    \section*{Приложение 1: Ход решения метода неопределённых коэффициентов}
    %    Выделим оранжевым исходы $f(\dots) = 0$, вычеркнем жёлтым те коэффициенты, которые встречаются в оранжевых строках, остальные выделим фиолетовым.
    %}
]{../docx/title.pdf}


\section*{Аннотация}

В данном документе излагается программный код программного обеспечения для эмуляции процессора «CEMUS» (эмулятора). 

Эмулятор представляет собой программу с графическим интерфейсом, разработанную на высокоуровневом условно интерпретируемом универсальном языке программирования со статической типизацией C\# (версия языка на момент разработки: 7.0) с использованием описанных далее сторонних библиотек.

Эмулятор позволяет продемонстрировать работу микропрограммного уровня процессора «CEMUS», настройку при помощи дешифратора, а также программного уровня процессора «CEMUS». Также при разработке в эмулятор закладывался функционал по набору программных текстов для процессора «CEMUS» - пригодных для загрузки в долговременную энергонезависимую память процессора. 

\newpage

\tableofcontents

\newpage

\getcode{App.xaml}
\getcode{App.xaml.cs}
\getcode{AssemblyInfo.cs}
\getcode{Cemus.cs}
\getcode{CemusCore.cs}
\getcode{CollectionUtils.cs}
\getcode{CommandConvertHelper.cs}
\getcode{CommandState.cs}
\getcode{CommandWrap.cs}
\getcode{CommentaryOpenCloser.cs}
\getcode{Configurations.cs}
\getcode{ContextDrawers.cs}
\getcode{Decoder.cs}
\getcode{DecoderHelper.cs}
\getcode{DecoderProcessors.cs}
\getcode{DecoderSingleMapping.cs}
\getcode{DialogHelper.cs}
\getcode{EmulationMode.cs}
\getcode{Emulator.cs}
\getcode{EmulatorEvents.cs}
\getcode{EventUtils.cs}
\getcode{ExecutionStyle.cs}
\getcode{FileSearchHelper.cs}
\getcode{FlagDrawHelper.cs}
\getcode{FlagNames.cs}
\getcode{FlowResolver.cs}
\getcode{FullProjectHelper.cs}
\getcode{GeneratedJsonSchema.cs}
\getcode{GlobalConfig.cs}
\getcode{IColorful.cs}
\getcode{ICommand.cs}
\getcode{INamedSaveOpener.cs}
\getcode{IResetable.cs}
\getcode{InterruptManager.cs}
\getcode{InterruptSources.cs}
\getcode{JsonedCfgOpener.cs}
\getcode{JsonedConfigs.cs}
\getcode{ListRowFabric.cs}
\getcode{MMURequest.cs}
\getcode{MacroCommand.cs}
\getcode{MacroContext.cs}
\getcode{MacroOpenCloser.cs}
\getcode{MacroTextBoxes.cs}
\getcode{MainWindow.xaml}
\getcode{MainWindow.xaml.cs}
\getcode{MemoryProvider.cs}
\getcode{MicroByteStack.cs}
\getcode{MicroCommand.cs}
\getcode{MicroCommandGetters.cs}
\getcode{MicroCommandsFields.cs}
\getcode{MicroContext.cs}
\getcode{MicroCore.cs}
\getcode{MicroEmulatorProgramTweakers.cs}
\getcode{MicroOpenCloser.cs}
\getcode{MicroStepResult.cs}
\getcode{ModelValue.cs}
\getcode{NumberUtils.cs}
\getcode{PageSwitchers.cs}
\getcode{PiplineStates.cs}
\getcode{PostScript.cs}
\getcode{ProgramList.cs}
\getcode{ProjectEncoder.cs}
\getcode{ProjectOpenerJSON.cs}
\getcode{QuickDecypher.cs}
\getcode{QuickDecypherExplanationBuilder.cs}
\getcode{RegWrapperAsFlags.cs}
\getcode{RegisterFile.cs}
\getcode{RegisterHelpDrawer.cs}
\getcode{RequestDialog.cs}
\getcode{Resources.Designer.cs}
\getcode{SaveAndOpen.cs}
\getcode{Settings.Designer.cs}
\getcode{ShiftUtils.cs}
\getcode{SimulationKeys.cs}
\getcode{SingleUseAlu.cs}
\getcode{StackHelper.cs}
\getcode{StateShepherd.cs}
\getcode{Terminal.cs}
\getcode{TextEditor.cs}
\getcode{TextHelps.cs}
\getcode{Trivia.cs}
\getcode{VisibilityState.cs}
\getcode{WindowListViews.cs}
\getcode{YamlParse.cs}
\getcode{createJsonSchema.cs}
\getcode{uCommandValidator.cs}
\getcode{uSimBackConverter.cs}




\newpage

\includepdf
[
pages={2}, 
    pagecommand={
    %    \section*{Приложение 1: Ход решения метода неопределённых коэффициентов}
    %    Выделим оранжевым исходы $f(\dots) = 0$, вычеркнем жёлтым те коэффициенты, которые встречаются в оранжевых строках, остальные выделим фиолетовым.
    }
]{../docx/title.pdf}

\end{document}
